\begin{frame}[fragile]{Tracking: Repository}
  \begin{columns}[t]
    \begin{column}{0.5\textwidth}
      Getting a Git Repository
      \begin{itemize}
        \item Make a Git repository
        \begin{minted}[
            breaklines,
            frame=single,
            fontsize=\footnotesize
          ]{bash}
git init [directory]
        \end{minted}
        \item Clone a Git repository
        \begin{minted}[
            breaklines,
            frame=single,
            fontsize=\footnotesize
          ]{bash}
git clone <repository> [<directory>]
        \end{minted}
      \end{itemize}
      \begin{center}
        (\href{https://git-scm.com/doc}{Link to Documentation})
      \end{center}
    \end{column}
    \begin{column}{0.5\textwidth}
      Subdirectory: .git
      \begin{itemize}
        \item Contains repository files
        \begin{flushleft}
          \footnotesize
Every version of every file for the history of the project is pulled down by
default when you run git clone.
        \end{flushleft}
        \item The skeleton for the repository
        \begin{flushleft}
          \footnotesize
The source of magic. What are contained in the .git directory? 
(\href{https://git-scm.com/book/en/v2/Git-Internals-Git-Objects}{Git Internals})
        \end{flushleft}
      \end{itemize}
    \end{column}
  \end{columns}
\end{frame}
